\documentclass{eajam}
%%%%% journal  info %%%%%%%%%
\setcounter{page}{1}
\renewcommand\thisnumber{x}
\renewcommand\thisyear {200x}
\renewcommand\thismonth{xxx}
\renewcommand\thisvolume{xx}
%%%%%%%%   end  %%%%%%%%%%%%%

\usepackage{charter}
\usepackage[charter]{mathdesign}

%%%%% author macros %%%%%%%%%
% place your own macros HERE
%%%%% end %%%%%%%%%

\begin{document}

%%%%% title and author(s):
% \markboth{Author(s)}{Short Title}
% \title{Title}

\markboth{Authors}{Short Title}
\title{Here is the Title}

% single author:
% \author[AUTHOR]{AUTHOR\corrauth}
% \address{address of AUTHOR}
% \email{{\tt email address of AUTHOR} (AUTHOR)}
\author[Only Author]{Only Author\corrauth}
\address{School of Mathematical Sciences, Beijing International University,
Beijing 12345, China.}
\email{{\tt email@address} (Only Author)}

% multiple authors:
% Please mark \corrauth after the name of the corresponding author.
% different addresses:
%\author[AUTHOR1 and AUTHOR2]{AUTHOR1\affil{1}\comma\corrauth and AUTHOR2\affil{2}}
%\address{\affilnum{1}\ address of AUTHOR1\\
%\affilnum{2}\ address of AUTHOR2}
%
%same address:
%\author[AUTHOR1, AUTHOR2 and AUTHOR3]{AUTHOR1, AUTHOR2\corrauth and AUTHOR3}
%\address{address of AUTHOR1, AUTHOR2 and AUTHOR3}
%
%\emails{{\tt email of AUTHOR1} (AUTHOR1), {\tt email of AUTHOR2} (AUTHOR2), {\tt email of AUTHOR3} (AUTHOR3)}
%

%%%%% Begin Abstract %%%%%%%%%%%
\begin{abstract}
The abstract should provide a brief summary of the main findings of
the paper. The abstract should not be too long. Normally, it should
not be longer than half page.
\end{abstract}
%%%%% end %%%%%%%%%%%

%%%%% Keywords %%%%%%%%%%%
\keywords{Moving mesh method, conservative interpolation, iterative
method, $l^2$ projection.}

%%%% AMS subject classifications %%%%
\ams{65M10, 78A48}

%%%% maketitle %%%%%
\maketitle

%%%% Start %%%%%%
\section{Introduction}
\label{sec1} In the past two decades, there has been important
progress in developing adaptive mesh methods for PDEs. Mesh
adaptivity is usually of two types in form: local mesh refinement
and moving mesh method.

\section{Preparation of manuscript}
\label{sec2} The Title Page should contain the article title,
authors' names and complete affiliations, footnotes to the title,
and the postal address for manuscript correspondence (including
e-mail address and fax numbers). The Abstract should provide a brief
summary of the main findings of the paper.

References should be cited in the text by a number in square brackets. Literature cited should appear on a
separate page at the end of the article and should be styled and punctuated using standard abbreviations for
journals (see Chemical Abstracts Service Source Index, 1989). For unpublished lectures of symposia,
include title of paper, name of sponsoring society in full, and date. Give titles of unpublished reports
with "(unpublished)" following the reference. Only articles that have been published or are in press
should be included in the references. Unpublished results or personal communications should be cited
as such in the text.
Please note the sample at the end of this paper.

Equations should be typewritten whenever possible and the number
placed in parentheses at the right margin. Reference to equations
should use the form "Eq.~(1.1)" or simply "(1.1)." Superscripts and
subscripts should be typed or handwritten clearly above and below
the line, respectively.

Figures should be in a finished form suitable for publication. Number figures consecutively with Arabic numerals.
Lettering on drawings should be of professional quality or generated by high-resolution computer graphics and
must be large enough to withstand appropriate reduction for publication.


%%%% Acknowledgments %%%%%%%%
\section*{Acknowledgments}
The author would like to thank the referees for the helpful
suggestions.

%%%% Bibliography  %%%%%%%%%%
\begin{thebibliography}{99}
\bibitem{Berger}{\sc M. J. Berger and P. Collela}, {\em Local adaptive mesh refinement
for shock hydrodynamics}, J. Comput. Phys., 82 (1989), pp. 62--84.
\bibitem{deBoor}{\sc C. de Boor}, {\em Good approximation by splines with variable knots II}, in Springer Lecture
 Notes Series 363, Springer-Verlag, Berlin, 1973.
\bibitem{TanTZ} {\sc Z. J. Tan, T. Tang, and Z. R. Zhang}, {\em A simple moving mesh method for one- and
two-dimensional phase-field equations}, J. Comput. Appl. Math., to
appear.
\bibitem{Toro}{\sc E. F. Toro}, {\em Riemann Solvers and Numerical Methods for Fluid
Dynamics}, Springer-Verlag, Berlin Heidelbert, 1999.
\end{thebibliography}

\end{document}
